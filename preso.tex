\documentclass[xetex]{beamer}

\usepackage{xltxtra}
\usepackage{xcolor}
\usepackage{pgfplots}

\usetheme{Rochester}
\beamertemplatenavigationsymbolsempty

\title{cuRAND vs OpenCL}
\author{Dányi Bence\\ Kálmán Viktor}

\begin{document}
  \frame{\titlepage}
  \begin{frame}
    \frametitle{cuRAND}
    Az NVIDIA megoldása véletlen számok generálására
    \begin{itemize}
      \item Host API
      \item Device API
      \item Pszeudorandom sorozat: statisztikailag véletlen számok
      \item Kvázirandom sorozat (alacsony diszkrepanciájú): egy $n$-dimenziós teret egyenletesen töltenek ki
      \item Egyenletes/normál/poisson eloszlás
    \end{itemize}
  \end{frame}
  \begin{frame}
    \frametitle{Host API}
    A cuRAND API hívható hoszt környezetben is
    \begin{itemize}
      \item $n$ szám generálása (párhuzamosan)
      \item fallback CPU generátorra, ha nem érhető el CUDA
      \item minden szükséges paraméter egy \texttt{curandGenerator\_t}-ben
    \end{itemize}
  \end{frame}
  \begin{frame}
    \frametitle{Pszeudorandom sorozat}
    \begin{tikzpicture}
      \begin{axis}[scatter/classes={a={mark=o,draw=black}}]
        \addplot[scatter,only marks,scatter src=explicit symbolic,row sep=crcr] table[meta=label,col sep=comma] {pseudo-points.csv};
      \end{axis}
    \end{tikzpicture}
  \end{frame}
  \begin{frame}
    \frametitle{Alacsony diszkrepanciás sorozat}
    \begin{tikzpicture}
      \begin{axis}[scatter/classes={a={mark=o,draw=black}}]
        \addplot[scatter,only marks,scatter src=explicit symbolic,row sep=crcr] table[meta=label,col sep=comma] {quasi-points.csv};
      \end{axis}
    \end{tikzpicture}
  \end{frame}
\end{document}
